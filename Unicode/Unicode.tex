% author:	 sanfusu
% contact:	 sanfusu@foxmail.com

\documentclass[UTF8]{ctexart}

\ctexset {
	section={
		name={第, 节},
	}
}


\begin{document}
	\title{Unicode}
	\author{sanfusu}
	\maketitle
	
	\begin{abstract}
		Unicode 说来只是一种编码而已,但其具体的编码方式一直困惑于心。UTF-8、UTF-16 等不同术语的区别也是从未做过详细的区分,毕竟人们乐于拿来主义。
		本文重点解答关于 Unicode 的各种疑惑。
	\end{abstract}
	
	\section{术语解释}
	\begin{itemize}
		\item [UCS]{通用字符集}

	\end{itemize}

	\section{目的}
	Unicode 用于编码底层字符\footnote{基本书写单位,而非各种自行变体,可以理解为字体}。在文本处理过程中,Unicode 也只是提供单一的编码点\footnote{实际上只是一个数字或者编号,和 ASCII 编码类似的概念,实际渲染交由软件处理}。

	\section{UTF-8}
	1 \textasciitilde{} 4 个字节。一个字节用于 ASCII 的 128 个编码点。
	
	\section{USC-2}
	USC-2 是 UTF-16 的前身,使用两个字节来表示 Unicode 中的前 65536 编码点,也被称为 BMP\footnote{Basic Multilingual Plane,基础多语言面板} 。但事实上目前 Unicode 一共有 17 个板块,共 1114112 个编码点,远超过 USC-2 所能表示的范围。
		
	\section{UTF-16}
	能够编码 Unicode 中所有的编码点。UTF-16 起源于 UCS-2 (固定 16 bit 长度的通用字符集,即两个字节),但由于 16 bit 不足以容纳全部的编码点,故采用了可变长度的编码方式。	
	每一个编码点可以由一或两个 16-bit 的编码单元组成,也就是 2 \~{} 4个字节。
	
	UTF-16 中的 16 bit 部分和 UCS-2 保持一致,4 字节编码中用于其他面板,但不能包含 U+D800 -- U+DFFF 范围内的编码点\footnote{问题:该保留范围出现的意义?}。
	
	UTF-16 编码主要用于 Windows 和一些编程语言(Java、JavaScript),Unix 类系统中用的较少。
	WHATWG 因安全问题建议避免使用 UTF-16。
	

	
	
	
\end{document}