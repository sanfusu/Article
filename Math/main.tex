\documentclass{ctexart}
\usepackage{textcomp, listings, xcolor-material, enumitem, tabularx,amsmath,multirow, latexsym, amssymb}
\usepackage[noend]{algpseudocode}
\usepackage{algorithm, algorithmicx}
\usepackage[hmargin=1.25in, vmargin=1in]{geometry}

\usepackage[amsmath,hyperref,amsthm]{ntheorem}
%\usepackage[T1]{fontenc}
\usepackage{hyperref}

\setlist{noitemsep}
\setlist[1]{
labelindent=\parindent,
leftmargin=*
}
\setlist[description]{font=\ttfamily\bfseries}

\hypersetup{
	colorlinks=true,
	allcolors=MaterialCyan,
	anchorcolor=MaterialCyan,
	pdfauthor=sanfusu,
	pdfkeywords=golang,
}

\ctexset {
%	chapter/format += \raggedleft,
	section/name = {\dag{},},
%	section/number = \Roman{section},
	section/format += \raggedright,
	subsection/name = {\textperiodcentered{},},
%	subsection/number=\arabic{subsection},
%	subsection/format += \centering,
}

\newtheorem{Question}{题}[section]
%\theoremstyle{definition} \newtheorem{Definition}{定义}[section]


\title{数学证明笔记}
\author{sanfusu $<$\href{mailto:sanfusu@foxmail.com}{sanfusu@foxmail.com}$>$}
\date{\today}
\begin{document}
\maketitle

\section{集合}
\begin{Question}
证明集合 $T=\{a_1, a_2, a_3, \ldots, a_n\}$ 共有 $2^n$ 个子集。
\end{Question}
\begin{proof}
设集合 $T_1 \subset T$
$\forall a_k \in T$,均有两种情况,$a_k \in T_1$ 或 $a_k \notin T_1$。可用 0 和 1 来编码这两个情况,并将 T 中的元素按下标排序,可形成一个二进制 bit 序列组成的数。该数的范围为 $[0, 2^{n-1}]$,即 $2^n$ 个数字(代表 $2^n$ 个子集)。
\end{proof}

\begin{Question}
证明有理数集 $Q$ 是可列集。
\end{Question}
\begin{Question}
证明:
\begin{enumerate}
\item 任意无限集比包含一个可列子集
\item 设 $A$ 与 $B$ 都是可列子集,证明 $A \cup B$ 也是可列子集。
\end{enumerate}
\end{Question}
%
%\begin{Remark}
%sf
%\end{Remark}

\end{document}