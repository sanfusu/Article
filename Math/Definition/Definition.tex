\documentclass[UTF8]{ctexart}
    \usepackage{amsmath}
\usepackage[thmmarks,thref,hyperref,standard]{ntheorem}
\usepackage{ulem}
\usepackage{xcolor}
\usepackage{hyperref} % 尽量确保是最后一个加载包
\hypersetup{
    pdfauthor=sanfusu,
    bookmarks=true
    colorlinks=true,
    pdfborder={0,0,0}
}
% \theoremstyle{definition} \newtheorem{def}{定义}[section]
% \theoremnumbering{fnsymbol}
% \theoremstyle{change}
% \theoremheaderfont{\bfseries}
\theoremsymbol{{\scriptsize$\clubsuit$}}
\theoremseparator{:}
\renewtheorem{Definition}{Def.}

\theoremsymbol{{\scriptsize$\diamondsuit$}}
\theoremheaderfont{\bfseries}
\renewtheorem{Remark}{Note.}


\begin{document}
% \begin{titlepage}
    \title{定理及证明}
    \author{sanfusu}
    \date{\today}
    \maketitle
% \end{titlepage}

\section{数列}
\begin{Definition}
    无穷多个数的排列被称为数列,通常记为 ${a_n}$。
    这里的下标依次取遍正整数集 $N^*$。
\end{Definition}

\subsection{收敛数列}
\begin{Definition}
    设 ${a_n}$ 是一个数列,$a$ 是一个实数,对任意给定的 $\varepsilon > 0$,
    $\exists N \in N^*$ 使得对任意 $n>N$ 有
    \begin{equation*}
        |a_n-a|<\varepsilon,
    \end{equation*}
    则说数列${a_n}$ 当 $n$ 趋向于无穷大时以 $a$ 为极限,记为
    \begin{equation*}
        \lim_{n\to \infty} = a,
    \end{equation*}
    也可简记为 $a_n \to a (n \to \infty)$。我们也说数列 ${a_n}$ 收敛于 $a$。
    存在极限的数列称为收敛数列;不收敛的数列称为发散数列。
\end{Definition}
\begin{remark}
    数列极限为实数,但是否可以考虑扩展为复数域。
\end{remark}
\end{document}