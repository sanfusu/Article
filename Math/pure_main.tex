% !TEX root=./pure_mathematics.tex
\begin{document}
% \begin{titlepage}
    \title{纯数学}
    \author{sanfusu}
    \date{\today}
    \maketitle
% \end{titlepage}

\section{介绍}
纯数学的一个核心理论是一般性。使用一般性可以包括以下优势:
\begin{itemize}
    \item 一般性理论或数学结构可以导致对原始理论或结构更为深入的理解。
    \item 一般性可以简化演讲高,导致更为简短的证明或论证。
    \item 一般性可以避免重复劳动,证明一般性结果可以避免独立的证明几个分离的案例,或者可以直接使用其他领域的结果。
    \item 一般性可以促进不同数学分支的联系。
\end{itemize}

\subsection{纯数学的分支}

\paragraph{分析} 关心函数属性。处理连续、极限、微分、积分等概念,因此为牛顿和莱布尼兹提出的微积分提供坚实的基础。实分析研究实函数,复分析则将前面提及的理论扩展至复数领域。泛函分析则是研究无穷维数向量空间并将函数视为这些空间中的点。
\paragraph{抽象代数} 研究集合上定义的二元运算。
\paragraph{集合} 研究形状和空间。
\paragraph{数论} 研究正整数。
\paragraph{拓扑} 几何的当代扩展。
\end{document}