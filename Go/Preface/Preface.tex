% !TeX root = ../Main

\chapter{Preface}
%\section{目的以及本文档主旨}
\textbf{首先这一份关于 Golang 的文档,由最初的日常笔记发展而来。写字者本人并不熟悉 golang,这份册子也仅是为了学习而创建。}

\textbf{本文档并不旨在教会任何人关于 Golang 的任何东西,但会尽可能严谨的描述 Golang 的各部细节。目前本册子正在编写之中,任何内容都会随着时间的推移而补充。}

\section{综计}
\subsection{字符编码}
%Go 语言源文件使用 UTF-8 编码,这意味着一个字符不一定只有一个字节。参考 rune 字面值,详见 页\pageref{rune-literal}。

\paragraph{记号}
记号形成 Go 语言的词汇。有四种类型记号:
\begin{itemize}
	\item identifiers---标识符
	\item keywords---关键词
	\item operators---运算符
	\item punctuation---标点符
	\item literals---字面值
\end{itemize}

\paragraph{分号插入规则}
\begin{enumerate}
	\item  如果行末记号是以下几种情况,则自动插入:
	\begin{enumerate}
		\item 标识符
		\item 整型、浮点、虚数、rune或者字符串字面值
		\item  break, continue, fallthrough, return 中的一个关键词
		\item  \verb|++|, \verb|--|, \verb|)|, \verb|]|, 或者 \verb|}| 中的一个运算符和标点符
	\end{enumerate}
	\item  如果复杂语句占据单行,则可以在 \verb|)| 和 \verb|}| 之前省略分号
\end{enumerate}