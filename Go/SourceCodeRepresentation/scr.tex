% !TeX root = ../Main.tex

\chapter{源码表示}
原代码是 \href{http://en.wikipedia.org/wiki/UTF-8}{UTF-8} 编码格式的 Unicode 文本。
这些文本并不是正规化的,所以单个重音化的编码点与那些通过重音符和字母构成的相同字符并不一样,组合出来的字符被视为两个编码点。
简单的来说,本文档将使用未加任何限制的术语 \emph{character(字符)} 来指示源文本中的 Unicode 编码点。

区分每一个编码点,比如,大写字母和小写字母是不同的字符\footnote{按编码点来区分}。

实现上的限制:为了和其他工具兼容,编译器可能不允许源文本中出现 NUL\footnote{U+0000} 字符。

同样是实现上的限制:为了兼容其他工具,如果 UTF-8 的字节序记号(U+FEFF)位于源文本的第一个 Unicode 编码点,则编译器可能会忽视该记号。字节序记号可能会被禁止出现在源文本的其他任何地方。

\section{字符}
下列术语被用来表示特定的 Unicode 字符类:
\begin{lstlisting}[style=EBNF]
newline        = /* the Unicode code point U+000A */
unicode_char   = /* an arbitrary Unicode code point except newline */
unicode_letter = /* a Unicode code point classified as "Letter" */
@\label{unicodeDigit}@unicode_digit  = /* a Unicode code point classified as "Number, decimal digit" */
\end{lstlisting}
在 \href{http://www.unicode.org/versions/Unicode8.0.0/}{Unicode 8.0 标准},
第 4.5 节 ``General Category'' 中,定义了一组字符类别。
Go 将Lu,LI,Lt,Lm 或者 Lo 字母类别中的所有字符作为 Unicode 字母,
并将 Number 类别中的 Nd 作为 Unicode 数字。

\section{字母和数字}
下划线字符 \_(U+005F) 被视为一个字母。
\begin{EBNF}
letter        = unicode_letter | "_" .@\label{letter}@
decimal_digit = "0" … "9" .
octal_digit   = "0" … "7" .
hex_digit     = "0" … "9" | "A" … "F" | "a" … "f" .
\end{EBNF}



