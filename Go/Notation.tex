% !TeX root = Main.tex

\chapter{记号}
% @Note 脚注或许可以去除
语法格式通过扩展的巴克斯范式(EBNF)表示。
\begin{EBNF}
Production	= production_name "=" [ Expression ] "." .
Expression 	= Alternative { "|" ALternative } .
Alternative = Term { Term } .
Term        = production_name | token [ "…" token ] | Group | Option | Repetition .
Group       = "(" Expression ")" .
Option      = "[" Expression "]" .
Repetition  = "{" Expression "}" .
\end{EBNF}

Productions 是由 terms 以及下列操作符构成的表达式,按递增优先级如下:
\begin{description}[font=\ttfamily\bfseries, style=nextline, leftmargin=2\parindent, labelindent=\parindent]
	\item [|] alternation
	\item [()] grouping
	\item [{[]}] option (0 or 1 times)
	\item [\{\}] repetition (0 to n times)
\end{description}
小写的产生式名被用来标识词法记号。
非终止符使用峰驼命名法。
词法记号被包含在双引号 \code|""| 或者反撇号 \lstinline|``| 中。

 a \dots{} b 形式标识 a 到 b 的可替代集合。水平省略号 \dots{} 在本规范中用来表示各种枚举或代码片段。
 字符 \dots(与三个字符 $\ldots$ 相反)并不是 Go 语言中的记号。