% !TeX root = Main.tex
\chapter{类型和值的属性}

\section{类型特征}
两个类型要么相等要么不等。

一个定义的类型永远和其他任何类型不同。
非定义类型在其底层类型字面量结构相等时相等;
也就是说具有相同的字面结构以及相应的组件具有相同的类型。具体为:
\begin{itemize}
\item 如果两个数组类型具有相同的元素类型和数组长度,则该两个数组相等。
\item 如果两个切片类型具有相同的元素类型,则相等。
\item 如果两个结构体具有相同的字段序列,并且相应的字段具有相同的名字和相等的类型以及相等的标签。不同包中的未导出的字段名永远不相同。
\item 两个指针如果具有相等的基类型,则相等。
\item 如果两个函数类型具有相同数量的参数和结果值,并且相应的参数和结果类型相等(不用管函数是不是可变参数函数),则两个函数类型相等。参数和结果名不需要匹配。
\item 两个接口类型在其具有相同名字的方法集和等价的函数类型时相等。不同包中的非导出方法名永远不同。方法顺序不相干。
\item 两个 map 类型在其具有相等的 key 和元素类型时相等。
\item 两个通道类型在他们具有相等的元素类型和相同的方向时相等。
\end{itemize}

以下声明中
\begin{lstlisting}[style=golang]
type (
	A0 = []string
	A1 = A0
	A2 = struct{ a, b int }
	A3 = int
	A4 = func(A3, float64) *A0
	A5 = func(x int, _ float64) *[]string
)

type (
	B0 A0
	B1 []string
	B2 struct{ a, b int }
	B3 struct{ a, c int }
	B4 func(int, float64) *B0
	B5 func(x int, y float64) *A1
)

type	C0 = B0
\end{lstlisting}
等价的类型为:
\begin{lstlisting}[style=golang]
A0, A1, and []string
A2 and struct{ a, b int }
A3 and int
A4, func(int, float64) *[]string, and A5

B0 and C0
[]int and []int
struct{ a, b *T5 } and struct{ a, b *T5 }
func(x int, y float64) *[]string, func(int, float64) (result *[]string), and A5
\end{lstlisting}
\lstinline|B0| 和 \lstinline|B1| 是通过不同的类型定义创建的新类型,因此这两个类型并不相同;
\lstinline[style=golang]|func (int, float64) *B0| 和 \lstinline[style=golang]|func(x int, y float64) *[]string| 不同是因为 \lstinline|B0| 不同于 \lstinline[style=golang]|[]string|。



\section{可赋值性}
如果满足以下条件,则值 \lstinline|x| 可以赋值给类型 \lstinline|T| 的变量(``\lstinline|x| 是可赋值给 \lstinline|T| 的''):
\begin{itemize}
\item \lstinline|x| 的类型和 \lstinline|T| 完全相同。
\item \lstinline|x| 的类型 \lstinline|v| 和 \lstinline|T| 具有相同的底层类型,并且 \lstinline|v| 和 \lstinline|T| 中至少有一个不是定义的类型。
\item \lstinline|T| 是一个借口类型,并且 \lstinline|x| 实现了 \lstinline|T|。
\item \lstinline|x| 是一个双向通道值,\lstinline|T| 是一个通道类型,\lstinline|x| 的类型 \lstinline|V| 和 \lstinline|T| 具有完全相同的元素类型,并且这两个类型中至少有一个不是定义的类型。
\item \lstinline|x| 是预声明标识符 \lstinline|nil| 并且 \lstinline|T| 是一个指针,函数,切片,图,通道,或者借口类型。
\item \lstinline|x| 是一个可被类型 \lstinline|T| 表示的无类型常量。
\end{itemize}


\section{可表达性}
当满足下列条件之一时,常量 \lstinline|x| 可被表示为类型 \lstinline|T| 的值:
\begin{itemize}
\item \lstinline|x| 在由 \lstinline|T| 决定的值得集合中。
\item  \lstinline|T| 是一个浮点类型,并且 \lstinline|x| 可以无溢出的四舍五入到 \lstinline|T| 的精度范围内。使用 IEEE 754 的偶数约算规则进行四舍五入,但是 IEEE 的非负零简化成一个无符号零。注意常量值永远不可能是一个 IEEE 非负零,NAN,或者无穷。
\item \lstinline|T| 是一个复数类型,\lstinline|x| 的组件 \lstinline|real(x)| 和 \lstinline|imag(x)| 可以由 \lstinline|T| 的组件类型(\lstinline|float32| 或 \lstinline|float64| 的值表示。
\end{itemize}
\begin{table}
\begin{tabularx}{\textwidth}{llX}
\lstinline|x|		&\lstinline|T|&\lstinline|x| 能够被 \lstinline|T| 的值表示的原因是 \\
\hline
\lstinline|'a'|		&	byte	&	97 在字节类型的值得集合中 	\\
97          		&	rune	&	\lstinline|rune| 是 \lstinline|int32| 的别名,并且 97 在 32-bit 整型值得集合中 \\
\lstinline|"foo"|   &	string	&	\lstinline|"foo"| 在字符串值得集合中\\
1024            	&	int16	&	1024 在16-bit 整型值的集合中 \\
42.0            	&	byte	&	42 在无符号 8-bit 整型值得集合中 \\
1e10            	&	uint64 	&	10000000000 在 64-bit 无符号整型值的集合中 \\
2.718281828459045 	&	float32	&	2.718281828459045 四舍五入到 2.7182817 后再 float32 值得集合中。  \\
-1e-1000    		&	float64	&   -1e-1000 四舍五入到 IEEE -0.0 进一步简化到 0.0 \\
0i                 	&	int		&    0 是一个整型值	\\
(42 + 0i)          	&	float32	&   42.0 (虚部为 0) 在 float32 值得集合中 \\

\\

\lstinline|x|	&	\lstinline|T|	&	\lstinline|x| 无法被 \lstinline|T| 表示的原因 \\
\hline
0				&	bool			& 	0 不在布尔类型值的集合中	\\
\lstinline|'a'| &	string			&	\lstinline|'a'| 是一个 rune,不在字符串值的集合中	\\
1024			&	byte			& 	1024 不在无符号 8-bit 整型值的集合中	\\
-1				&	uint16			&	-1 不在无符号 16-bit 整型值的集合中	\\
1.1				&	int				&	1.1 不是一个整型值	\\
42i				&	float32			&	\lstinline|0 + 42i| 不在 float32 值的集合中	\\
1e1000			&	float64			&	1e1000	在约数后溢出为 IEEE 的正无穷(+Inf)\\

\end{tabularx}
\end{table}






