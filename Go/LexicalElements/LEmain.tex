% !TeX root = ../Main.tex

\chapter{Lexical elements}

\section{注释}
注释充当程序文档。有两种形式:
\begin{enumerate}
    \item 行注释以字符序列 \verb|//| 开头,并且结束于行尾。
    \item 通用注释起始于 \verb|/*| 并且结束于随后所遇到的第一个字符序列 \verb|*/|。
\end{enumerate}
注释不能起始于一个 rune 或者字符串文字,或者一个注释内部。
不包含换行的通用注释和空格的作用一样。
任何其他注释都和换行的作用一致。

\section{记号}
记号形成 Go 语言的词汇。
有四种类别的记号:\emph{标识符},\emph{关键词},\emph{操作符}和\emph{标点符},以及\emph{字面量}。
\emph{White space} 由 空格 (U+0020),水平制表 (U+0009),回车 (U+000D) 和换行 (U+000A) 组成,
如果空白字符没起到分割记号的作用,则将会被忽视。
另外,换行和文件结尾会触发分号的插入。
当输入分解为记号时,下一个记号将会是形成有效记号的最长字符序列。

\section{分号}
形式化语法使用分号 ``;'' 作为产生式的终止符。
Go 程序可以通过以下两个规则省略大部分分号:
\begin{enumerate}
\item 当输入分解为语言符号时,如果一行的最后一个语言符号为以下几种情况,则会自动插入到行末。
	\begin{itemize}
		\item 标志符
		\item 整型, 浮点,虚数,rune,或者字符串文字
		\item 关键词 \verb|break|, \verb|continue|, \verb|fallthrough|, 或者 \verb|return| 中的一个。
		\item \verb|++, --, ), ]| 或 \verb|}| 运算符和标点符号中的一个。
	\end{itemize}
\item 为了能让复合语句占据单行,\verb|)| 和 \verb|}| 前的分号可以省略。 
\end{enumerate}
为了反映惯用的用法,本文档中的代码会使用以上规则来省略掉分号。


\section{标识符}
标识符命名程序条目,比如变量和类型。一个标识符是一个或者多个字母和数字的序列。
标识符中的第一个字符必须是一个字母。
\begin{lstlisting}[style=EBNF]
identifier = letter { letter | @\hyperref[unicodeDigit]{unicode\_digit}@ } .
\end{lstlisting}

\begin{lstlisting}[style=golang]
a
_x9
ThisVariableIsExported
@$\alpha\beta$@
\end{lstlisting}
部分标识符是预先声明的。

\section{关键词}
以下保留的关键词,并且不能用作标识符。
\begin{lstlisting}[style=golang]
break        default      func         interface    select
case         defer        go           map          struct
chan         else         goto         package      switch
const        fallthrough  if           range        type
continue     for          import       return       var
\end{lstlisting}

\section{运算符和标点符号}
以下字符序列标识运算符(包括赋值操作符)以及标点符号:
\begin{lstlisting}[style=golang]
+    &     +=    &=     &&    ==    !=    (    )
-    |     -=    |=     ||    <     <=    [    ]
*    ^     *=    ^=     <-    >     >=    {    }
/    <<    /=    <<=    ++    =     :=    ,    ;
%    >>    %=    >>=    --    !     ...   .    :
     &^          &^=
\end{lstlisting}

\section{整型字面量}
一个整型字面量是标识整型常量的一个数字序列。
可选前缀可以设置一个非十进制基:0 为八进制,0x 或者 0X 为十六进制。
在十六进制字面量中,字母 a-f 和 A-F 代表 10 到 15。
\begin{lstlisting}[style=EBNF]
int_lit     = decimal_lit | octal_lit | hex_lit .
decimal_lit = ( "1" … "9" ) { decimal_digit } .
octal_lit   = "0" { octal_digit } .
hex_lit     = "0" ( "x" | "X" ) hex_digit { hex_digit } .
\end{lstlisting}

\begin{lstlisting}[style=golang]
42
0600
0xBadFace
170141183460469231731687303715884105727
\end{lstlisting}


\section{浮点字面量}
浮点字面量是浮点常量的十进制表示。
其具有一个整数部分,一个十进制小数点,一个小数部分,和一个指数部分。
整数和小数部分有十进制数组成;
指数部分为 e 或者 E 后面紧跟着一个可选的有符号十进制指数。
可以省略整数部分或者小数部分中的一个;
类似小数点或者指数中的一个可以被省略。
\begin{lstlisting}[style=golang]
float_lit = decimals "." [ decimals ] [ exponent ] |
            decimals exponent |
            "." decimals [ exponent ] .
decimals  = decimal_digit { decimal_digit } .
exponent  = ( "e" | "E" ) [ "+" | "-" ] decimals .
\end{lstlisting}

\begin{lstlisting}[style=golang]
0.
72.40
072.40  // == 72.40
2.71828
1.e+0
6.67428e-11
1E6
.25
.12345E+5
\end{lstlisting}

\section{虚数字面量}
虚数字面量是复数常量虚部的十进制表示。
他由后面跟着字母 i 的浮点字面量或十进制整数组成。
\begin{lstlisting}[style=EBNF]
imaginary_lit = (decimals | float_lit) "i" .
\end{lstlisting}

\begin{lstlisting}[style=golang]
0i
011i  // == 11i
0.i
2.71828i
1.e+0i
6.67428e-11i
1E6i
.25i
.12345E+5i
\end{lstlisting}

% !TeX root = ../Main.tex


\section{Rune literals}
一个\rune{}字面量表示一个\rune{}常量,
一个用来标识 Unicode 编码点的整数值。
一个\rune{}字面量使用单引号包含一个或多个字符来表达,如 \lstinline|'x'| 或者 \lstinline|'\n'|。
除换行和未转义的单引号外,任何字符都可以出现在单引号内。
使用单引号包含的字符代表了字符本身的 Unicode 值,
但是以反斜杠开头的多字符序列会以各种格式对值进行编码。

\rune{}字面量的最简单形式是在引号内表示单个字符;
由于 Go 源文本是 UTF-8 编码的 Unicode 字符,所以多个 UTF-8 编码的字节可以使用单个整数值表示。
比如字面值 \lstinline|'a'| 拥有表示字面量 \lstinline|a| 的单个字节,
Unicode 编码点为 U+0061,值为 \lstinline|0x61|,但是 \lstinline|'ä'|则拥有两个字节,用来表示 a 的分音符号,
U+00E4,值为 \lstinline|0xe4|。

有几个反斜杠转义允许将任意值编码为 ASCII 文本。
有四种方式将整型值表示为数字常量:
\lstinline|\x| 后面跟上两个十六进制数;
\lstinline|\u| 后面跟上四个十六进制数;
\lstinline|\U| 后面跟上八个十六进制数;
\lstinline|\|  后面跟上三个八进制数。
每一种表示方法中,字面量的值为数字使用相应的进制表示的值。

尽管这些表示的结果均为整型,但是他们拥有不同的有效范围。
八进制转义表示的值必须在 0 到 255 范围内(包含 0 和 255)。
十六进制转义通过构造来满足该条件。% @FIXME 语义不明
转义 \lstinline|\u| 和 \lstinline|\U| 表示 Unicode 编码点,因此他们所表示的部分值可能是非法的,特别是大于 0x10FFFF 的值以及 surrogate havles。

反斜杠后面跟上特定的单字符表示特殊的值:
\begin{lstlisting}
\a   U+0007 alert or bell
\b   U+0008 backspace
\f   U+000C form feed
\n   U+000A line feed or newline
\r   U+000D carriage return
\t   U+0009 horizontal tab
\v   U+000b vertical tab
\\   U+005c backslash
\'   U+0027 single quote  (valid escape only within rune literals)
\"   U+0022 double quote  (valid escape only within string literals)
\end{lstlisting}

在\rune{}字面量里的其余以反斜杠开始的序列均是非法的。
\begin{lstlisting}[style=EBNF]
rune_lit         = "'" ( unicode_value | byte_value ) "'" .
unicode_value    = unicode_char | little_u_value | big_u_value | escaped_char .
byte_value       = octal_byte_value | hex_byte_value .
octal_byte_value = `\` octal_digit octal_digit octal_digit .
hex_byte_value   = `\` "x" hex_digit hex_digit .
little_u_value   = `\` "u" hex_digit hex_digit hex_digit hex_digit .
big_u_value      = `\` "U" hex_digit hex_digit hex_digit hex_digit
                           hex_digit hex_digit hex_digit hex_digit .
escaped_char     = `\` ( "a" | "b" | "f" | "n" | "r" | "t" | "v" | `\` | "'" | `"` ) .
\end{lstlisting}

\begin{lstlisting}[style=golang]
'a'
'ä'
'本'
'\t'
'\000'
'\007'
'\377'
'\x07'
'\xff'
'\u12e4'
'\U00101234'
'\''         // rune literal containing single quote character
'aa'         // illegal: too many characters
'\xa'        // illegal: too few hexadecimal digits
'\0'         // illegal: too few octal digits
'\uDFFF'     // illegal: surrogate half
'\U00110000' // illegal: invalid Unicode code point
\end{lstlisting}
% !TeX root = ../Main.tex

\section{字符串字面量}
字符串字面量表示从相连的字符序列中所获取的字符串常量。
字符串字面量有两种形式:原始字符串字面量和已翻译的字符串字面量。

原始字符串字面量是两个反撇号之间的字符序列,比如 \lstinline|`foo`|。除了反撇号自身,任何字符均可以出现在反撇号里。
原始字符串字面量的值是由反撇号之间未翻译过(默认是 UTF-8 编码)的字符序列所组成的字符串;
值得注意的是,反斜杠没有特殊意义,并且字符串可以包含换行。
原始字符串字面量中的回车字符(\lstinline|'\r'|)会被舍弃。

翻译过的字符串字面量为双引号之间的字符序列,比如 \lstinline|"bar"|。
除了双引号自身和换行之外任何字符都可以出现在双引号内。
双引号之间的文本形成字面量的值,但其中的反斜杠转义会根据\rune{}字面量中的限制来解释(除了 \lstinline|\'| 非法,而 \lstinline|\"| 结果字符串中的合法之外)。
三个八进制数字(\lstinline|\nnn| 和两个十六进制数转义表示结果字符串中的个体字节,其他所有的转义均表示个体字符的 UTF-8 编码(也可能是多字节)。
因此字符串字面量里面的 \lstinline|\377| 和 \lstinline|\xFF| 表示值为 0xFF=255 的单字节,
而 \"y, \lstinline|\u00FF|, \lstinline|\U000000FF | 和 \lstinline|\xc3\xbf| 表示 UTF-8 编码的字符 U+00FF 的两个字节 \lstinline|0xc3 0xbf|。

\begin{lstlisting}[style=EBNF]
string_lit             = raw_string_lit | interpreted_string_lit .
raw_string_lit         = "`" { unicode_char | newline } "`" .
interpreted_string_lit = `"` { unicode_value | byte_value } `"` .
\end{lstlisting}

\begin{lstlisting}[style=golang]
`abc`                // same as "abc"
`\n
\n`					 // same as "\\n\n\\n"
"\n"
"\""                 // same as `"`
"Hello, world!\n"
"日本語"
"\u65e5本\U00008a9e"
"\xff\u00FF"
"\uD800"             // illegal: surrogate half
"\U00110000"         // illegal: invalid Unicode code point
\end{lstlisting}
下面示例均表示相同的字符串:
\begin{lstlisting}[style=golang]
"日本語"							// UTF-8 input text
`日本語`							// UTF-8 input text as a raw literal
"\u65e5\u672c\u8a9e"                    // the explicit Unicode code points
"\U000065e5\U0000672c\U00008a9e"        // the explicit Unicode code points
"\xe6\x97\xa5\xe6\x9c\xac\xe8\xaa\x9e"  // the explicit UTF-8 bytes
\end{lstlisting}
如果源代码将一个字符表示为两个编码点,比如音调符号和字母的组合形式,若放在\rune{}字面值中,则结果将会出错,若放置在字符串字面量中则会出现两个编码点。
