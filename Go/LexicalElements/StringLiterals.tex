% !TeX root = ../Main.tex

\section{字符串字面量}
字符串字面量表示从相邻的字符序列中所获取的字符串常量。
字符串字面量有两种形式:原始字符串字面量和解释型字符串字面量。

原始字符串字面量是两个反撇号之间的字符序列,比如 \lstinline|`foo`|。除了反撇号自身,任何字符均可以出现在反撇号里。
原始字符串字面量的值是由反撇号之间未翻译过(默认是 UTF-8 编码)的字符序列所组成的字符串;
值得注意的是,反斜杠没有特殊意义,并且字符串可以包含换行。
原始字符串字面量中的回车字符(\lstinline|'\r'|)会被舍弃。

解释型字符串字面量为双引号之间的字符序列,比如 \lstinline|"bar"|。
除了双引号自身和换行之外任何字符都可以出现在双引号内。
双引号之间的文本形成字面量的值,但其中的反斜杠转义会根据\rune{}字面量中的限制来解释(除了 \lstinline|\'| 非法,而 \lstinline|\"| 合法之外)。
三个八进制数字(\lstinline|\nnn| 和两个十六进制数转义表示结果字符串中的个体字节,其他所有的转义均表示个体字符的 UTF-8 编码(也可能是多字节)。
因此字符串字面量里面的 \lstinline|\377| 和 \lstinline|\xFF| 表示值为 0xFF=255 的单字节,
而 \"y, \lstinline|\u00FF|, \lstinline|\U000000FF | 和 \lstinline|\xc3\xbf| 表示 UTF-8 编码的字符 U+00FF 的两个字节 \lstinline|0xc3 0xbf|。

\begin{lstlisting}[style=EBNF]
string_lit             = raw_string_lit | interpreted_string_lit .
raw_string_lit         = "`" { unicode_char | newline } "`" .
interpreted_string_lit = `"` { unicode_value | byte_value } `"` .
\end{lstlisting}

\begin{lstlisting}[style=golang]
`abc`                // same as "abc"
`\n
\n`					 // same as "\\n\n\\n"
"\n"
"\""                 // same as `"`
"Hello, world!\n"
"日本語"
"\u65e5本\U00008a9e"
"\xff\u00FF"
"\uD800"             // illegal: surrogate half
"\U00110000"         // illegal: invalid Unicode code point
\end{lstlisting}
下面示例均表示相同的字符串:
\begin{lstlisting}[style=golang]
"日本語"								// UTF-8 input text
`日本語`								// UTF-8 input text as a raw literal
"\u65e5\u672c\u8a9e"                    // the explicit Unicode code points
"\U000065e5\U0000672c\U00008a9e"        // the explicit Unicode code points
"\xe6\x97\xa5\xe6\x9c\xac\xe8\xaa\x9e"  // the explicit UTF-8 bytes
\end{lstlisting}
如果源代码将一个字符表示为两个编码点,比如音调符号和字母的组合形式,若放在\rune{}字面值中,则结果将会出错,若放置在字符串字面量中则会出现两个编码点。
