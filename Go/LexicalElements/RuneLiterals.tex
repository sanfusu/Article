% !TeX root = ../Main.tex


\section{Rune literals}
一个\rune{}字面量表示一个\rune{}常量,
一个用来标识 Unicode 编码点的整数值。
一个\rune{}字面量使用单引号包含的一个或多个字符来表达,如 \lstinline|'x'| 或者 \lstinline|'\n'|。
除换行和未转义的单引号外,任何字符都可以出现在单引号内。
使用单引号包含的字符代表了字符本身的 Unicode 值,
但是以反斜杠开头的多字符序列会以各种格式对值进行编码。

\rune{}字面量的最简单形式是在引号内表示单个字符;
由于 Go 源文本是 UTF-8 编码的 Unicode 字符,所以多个 UTF-8 编码的字节可以使用单个整数值表示。
比如字面值 \lstinline|'a'| 拥有表示字面量 \lstinline|a| 的单个字节,
Unicode 编码点为 U+0061,值为 \lstinline|0x61|,但是 \lstinline|'ä'| 则拥有两个(0xc3, 0xa4)用来表示 a 的分音符号的字节,
U+00E4,值为 \lstinline|0xe4|。

有几个反斜杠转义允许将任意值编码为 ASCII 文本。
有四种方式将整型值表示为数字常量:
\lstinline|\x| 后面跟上两个十六进制数;
\lstinline|\u| 后面跟上四个十六进制数;
\lstinline|\U| 后面跟上八个十六进制数;
\lstinline|\|  后面跟上三个八进制数。
每一种表示方法中,字面量的值为数字使用相应的进制表示的值。

尽管这些表示的结果均为整型,但是他们拥有不同的有效范围。
八进制转义表示的值必须在 0 到 255 范围内(包含 0 和 255)。
十六进制转义通过构造来满足该条件。% @FIXME 语义不明
转义 \lstinline|\u| 和 \lstinline|\U| 表示 Unicode 编码点,因此他们所表示的部分值可能是非法的,特别是大于 0x10FFFF 的值以及 surrogate havles。

反斜杠后面跟上特定的单字符表示特殊的值:
\begin{lstlisting}
\a   U+0007 alert or bell
\b   U+0008 backspace
\f   U+000C form feed
\n   U+000A line feed or newline
\r   U+000D carriage return
\t   U+0009 horizontal tab
\v   U+000b vertical tab
\\   U+005c backslash
\'   U+0027 single quote  (valid escape only within rune literals)
\"   U+0022 double quote  (valid escape only within string literals)
\end{lstlisting}

在\rune{}字面量里的其余以反斜杠开始的序列均是非法的。
\begin{lstlisting}[style=EBNF]
rune_lit         = "'" ( unicode_value | byte_value ) "'" .
unicode_value    = unicode_char | little_u_value | big_u_value | escaped_char .
byte_value       = octal_byte_value | hex_byte_value .
octal_byte_value = `\` octal_digit octal_digit octal_digit .
hex_byte_value   = `\` "x" hex_digit hex_digit .
little_u_value   = `\` "u" hex_digit hex_digit hex_digit hex_digit .
big_u_value      = `\` "U" hex_digit hex_digit hex_digit hex_digit
                           hex_digit hex_digit hex_digit hex_digit .
escaped_char     = `\` ( "a" | "b" | "f" | "n" | "r" | "t" | "v" | `\` | "'" | `"` ) .
\end{lstlisting}

\begin{lstlisting}[style=golang]
'a'
'ä'
'本'
'\t'
'\000'
'\007'
'\377'
'\x07'
'\xff'
'\u12e4'
'\U00101234'
'\''         // rune literal containing single quote character
'aa'         // illegal: too many characters
'\xa'        // illegal: too few hexadecimal digits
'\0'         // illegal: too few octal digits
'\uDFFF'     // illegal: surrogate half
'\U00110000' // illegal: invalid Unicode code point
\end{lstlisting}