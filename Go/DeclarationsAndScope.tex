% !TeX root = Main.tex
\chapter{声明和作用域}
一个声明将一个非空白标识符绑定到一个常量、类型、变量、函数、标签或者包。
程序中的每一个标识符必须被声明。
相同块中不可以将重复声明一个标识符,并且标识符不可同时声明在文件块和包块中。

空白标识符可以和声明中的任何其他标识符一样使用,但是不会引入一个绑定,因而不是一个声明。
在包块中,标识符 \lstinline|init| 可能会用作 init 函数声明,和空白标识符一样也不会引入一个新的绑定。
\begin{lstlisting}[style=EBNF]
Declaration   = ConstDecl | TypeDecl | VarDecl .
TopLevelDecl  = Declaration | FunctionDecl | MethodDecl .
\end{lstlisting}
被声明的标识符的作用域是标识符表示特定常量,类型,变量,函数,标签,或包的源文本区间。

Go 使用块进行此法分区:
\begin{enumerate}[labelindent=\parindent,leftmargin=*]
\item 预声明标识符作用域为全局块。
\item 在最顶层(所有函数之外)声明的用来表示一个常量,类型,变量,或者函数(非方法)的标识符的作用域为包块。
\item 一个导入的包名的作用域为包含导入声明的文件的文件域。
\item 一个表示一个方法接收器,函数参数或结果变量的标识符的作用域为函数体。
\item 函数内声明的常量或变量标识符的作用域起始于 ConstSpec 或 Var Spec(ShortVarDecl 表示短变量声明) 结尾以及最内层包含快结尾。
\item 函数内声明的类型标识符作用域起始于 TypeSpec 中的标识符并结束于最内层包含块的结尾。
\end{enumerate}

一个区块中声明的标识符可以其内部区块重新声明。
但内部区块中声明的标识符在作用域范围内表示内部声明所声明的条目。

包子句部分并不是一个声明;包名不会出现在任何作用域内。
其目的只是为了将一些文件归纳于同一个包中,并且为导入声明指定默认的包名。

\section{标签作用域}
标签通过标签化语句声明,并用在 break,continue 和 goto 语句中。
定义一个从未用过的标签是非法的。
和其他标识符相反,标签无区块作用域并且不会和非标签的标识符冲突。
标签的作用域为声明所在的函数体内,并且不包含任何嵌套的函数体。

\section{空白标识符}
空白标识符通过下划线字符 \_ 表示。
空白标识符用作匿名占位符并在声明、操作数和赋值中具有特殊意义,

\section{预声明标识符}
下面标识符隐式的声明在全局区块中:
\begin{description}[style=nextline,leftmargin=3\parindent]
\item[Types:] 	
	{\ttfamily bool byte complex64 complex128 error float32 float64 int int8 \\ int16 int32 int64 rune string	uint uint8 uint16 uint32 uint64 \\ uintptr}
\item[Constants:] 
	{\ttfamily true false iota}
\item[Zero value:] 
	{\ttfamily nil}
\item[Functions:] 
	{\ttfamily append cap close complex copy delete imag len make new panic print \\ println real recover}
\end{description}


\section{导出的标识符}
一个标识符可以被导出,从而允许在其他包里访问。
满足以下条件的标识符会被导出:
\begin{enumerate}
\item 标识符名的首字符为大写的 Unicode 字母(Unicode 类别为 ``Lu'');并且
\item 标识符声明在包区块中或者是一个字段名或是一个方法名。
\end{enumerate}
所有其他标识符不会导出。

\section{标识符的唯一性}
在给定的标识符集合中,如果一个标识符不同于集合中任何其他标识符,则称之为唯一。
两个标识符的不同点在于拼写,或出现在不同的包中并且没有被导出。其他情况下,认为标识符相同。

\section{常量声明}
一个常量声明将一组标识符(常量的名字)绑定到一组常量表达式的值。
标识符的数量必须和表达式的数量相等,并且左边第 n 个标识符被绑定到右边第 n 个表达式的值。
\begin{lstlisting}[style=EBNF]
ConstDecl      = "const" ( ConstSpec | "(" { ConstSpec ";" } ")" ) .
ConstSpec      = IdentifierList [ [ Type ] "=" ExpressionList ] .

IdentifierList = identifier { "," identifier } .
ExpressionList = Expression { "," Expression } .
\end{lstlisting}
如果出现类型,则所有常量会采用指定的类型,并且表达式必须能够赋值给该类型。
如果省略类型,常量则会采用相应表达式的类型。
如果表达式的值为无类型常量,声明的常量依旧保留无类型,并且该常量标识符表示该常量值。
比如,若表达式是一个浮点字面量,常量标识符则会表示一个浮点常量,即便字面量的小数部分是 0。
\begin{lstlisting}[style=golang]
const Pi float64 = 3.14159265358979323846
const zero = 0.0        // untyped floating-point constant
const (
	size int64 = 1024
	eof        = -1  	// untyped integer constant
)
/* a = 3, b = 4, c = "foo", untyped integer and string constants */
const a, b, c = 3, 4, "foo"  
const u, v float32 = 0, 3    // u = 0.0, v = 3.0
\end{lstlisting}
在使用原括弧包含的 \lstinline|const| 声明列表中,除了第一个 ConstSpec 其余的表达式列表均可以省略。
这种空列表等价于前面第一个非空表达式列表和其类型(如果有的话)的文本替换。
故,省略表达式列表等于重复前一个列表。
标识符的数量必须等于前一个列表中表达式的数量。
连同 iota 常提供量产生器一起可以提供轻量级的序列值声明:
\begin{lstlisting}[style=golang]
const (
	Sunday = iota
	Monday
	Tuesday
	Wednesday
	Thursday
	Friday
	Partyday
	numberOfDays  // this constant is not exported
)
\end{lstlisting}

\section{Iota}
在常量声明中,预声明标识符 \lstinline|iota| 表示连续无类型常量。
\lstinline|iota| 的值为常量声明中相应的 ConstSpec 的索引(从零开始)。
可以用来构造一组相关的常量:
\begin{lstlisting}[style=golang]
const (
	c0 = iota  // c0 == 0
	c1 = iota  // c1 == 1
	c2 = iota  // c2 == 2
)

const (
	a = 1 << iota  // a == 1  (iota == 0)
	b = 1 << iota  // b == 2  (iota == 1)
	c = 3          // c == 3  (iota == 2, unused)
	d = 1 << iota  // d == 8  (iota == 3)
)

const (
	u         = iota * 42  // u == 0     (untyped integer constant)
	v float64 = iota * 42  // v == 42.0  (float64 constant)
	w         = iota * 42  // w == 84    (untyped integer constant)
)

const x = iota  // x == 0
const y = iota  // y == 0
\end{lstlisting}

根据定义,在同一个 ConstSpec 中的多个 \lstinline|iota| 具有相同的值:
\begin{lstlisting}[style=golang]
const (
	bit0, mask0 = 1 << iota, 1<<iota - 1  // bit0 == 1, mask0 == 0  (iota == 0)
	bit1, mask1                           // bit1 == 2, mask1 == 1  (iota == 1)
	_, _                                  //                        (iota == 2, unused)
	bit3, mask3                           // bit3 == 8, mask3 == 7  (iota == 3)
)
\end{lstlisting}
最后一个示例中利用了隐式的重复最后一个非空表达式列表。


\section{类型声明}
一个类型声明将一个标识符即类型名绑定到一个类型。
类型声明具有两种形式:别名声明和类型定义。
\begin{lstlisting}[style=golang]
TypeDecl = "type" ( TypeSpec | "(" { TypeSpec ";" } ")" ) .
TypeSpec = AliasDecl | TypeDef .
\end{lstlisting}

\subsection{别名声明}
一个别名声明将一个标识符绑定到一个给定的类型上。
\begin{lstlisting}[style=EBNF]
AliasDecl = identifier "=" Type .
\end{lstlisting}
在该标识符的作用域内,提供类型别名的服务。
\begin{lstlisting}[style=golang]
type (
	nodeList = []*Node  // nodeList and []*Node are identical types
	Polar    = polar    // Polar and polar denote identical types
)
\end{lstlisting}

\subsection{类型定义}
类型定义会创建一个和给定类型拥有相同底层类型和操作,但截然不同的类型,并且将一个标识符绑定到新创建的类型。
\begin{lstlisting}[style=golang]
TypeDef = identifier Type .
\end{lstlisting}
新类型被称为定义的类型。该类型不同于其他任何类型,包括它的创建来源。
\begin{lstlisting}[style=golang]
type (
	Point struct{ x, y float64 }  // Point and struct{ x, y float64 } are different types
	polar Point                   // polar and Point denote different types
)

type TreeNode struct {
	left, right *TreeNode
	value *Comparable
}

type Block interface {
	BlockSize() int
	Encrypt(src, dst []byte)
	Decrypt(src, dst []byte)
}
\end{lstlisting}
定义的类型可能会有与其相关联的方法。
不会集成任何与给定类型绑定方法,但是接口类型或聚合类型的元素的方法集不会更改:
\begin{lstlisting}[style=golang]
// A Mutex is a data type with two methods, Lock and Unlock.
type Mutex struct         { /* Mutex fields */ }
func (m *Mutex) Lock()    { /* Lock implementation */ }
func (m *Mutex) Unlock()  { /* Unlock implementation */ }

// NewMutex has the same composition as Mutex but its method set is empty.
type NewMutex Mutex

// The method set of PtrMutex's underlying type *Mutex remains unchanged,
// but the method set of PtrMutex is empty.
type PtrMutex *Mutex

// The method set of *PrintableMutex contains the methods
// Lock and Unlock bound to its embedded field Mutex.
type PrintableMutex struct {
	Mutex
}

// MyBlock is an interface type that has the same method set as Block.
type MyBlock Block
\end{lstlisting}

类型定义可能会被用来定义不同的布尔,数值或字符串类型并且与其关联一些方法:
\begin{lstlisting}[style=golang]
type TimeZone int

const (
	EST TimeZone = -(5 + iota)
	CST
	MST
	PST
)

func (tz TimeZone) String() string {
	return fmt.Sprintf("GMT%+dh", tz)
}
\end{lstlisting}

\section{变量声明}
一个变量声明可以创建一个或多个变量,绑定到相应的标识符,并且给出每一个变量的类型和一个初始值。
\begin{lstlisting}[style=EBNF]
VarDecl     = "var" ( VarSpec | "(" { VarSpec ";" } ")" ) .
VarSpec     = IdentifierList ( Type [ "=" ExpressionList ] | "=" ExpressionList ) .
\end{lstlisting}

\begin{lstlisting}[style=golang]
var i int
var U, V, W float64
var k = 0
var x, y float32 = -1, -2
var (
	i       int
	u, v, s = 2.0, 3.0, "bar"
)
var re, im = complexSqrt(-1)
var _, found = entries[name]  // map lookup; only interested in "found"
\end{lstlisting}
如果给出表达式列表,变量则会通过赋值规则使用表达式来初始化变量。
否则,每一个变量会被初始化为 0 值。

变量会优先使用给出的类型,否则使用初始化时所赋的值的类型。
如果值为无类型常量,则会先转换为其默认类型;
如果时一个无类型布尔值,则会先转换为 \lstinline|bool| 类型。
预声明的值 \lstinline|nil| 必须使用显式的类型来初始化一个变量。
\begin{lstlisting}[style=golang]
var d = math.Sin(0.5)  // d is float64
var i = 42             // i is int
var t, ok = x.(T)      // t is T, ok is bool
var n = nil            // illegal
\end{lstlisting}

受限制于实现:编译器可能会认为在函数体内声明一个从未被使用的变量是非法声明。

\section{短变量声明}\label{sec:shortVarDeclare}
短变量声明使用以下格式:
\begin{EBNF}
ShortVarDecl = IdentifierList ":=" ExpressionList .
\end{EBNF}
相当于常规无类型带初始化的变量声明的简写形式:
\begin{lstlisting}[style=EBNF]
"var" IdentifierList = ExpressionList .
\end{lstlisting}

\begin{lstlisting}[style=golang]
i, j := 0, 10
f := func() int { return 7 }
ch := make(chan int)
r, w := os.Pipe(fd)  // os.Pipe() returns two values
_, y, _ := coord(p)  // coord() returns three values; 
					 // only interested in y coordinate
\end{lstlisting}

和常规变量声明不同,短变量声明可以使用相同的类型重新声明在之前同一块中声明过的变量(或者块为函数体的参数列表),并且在重声明时至少有一个非空白变量是新的。
相应的,重声明只能出现在多变量的短声明中。
重声明不会引入新的变量,他只会将新值赋值给原有的变量中。
\begin{lstlisting}
field1, offset := nextField(str, 0)
field2, offset := nextField(str, offset)  // redeclares offset
a, a := 1, 2	// illegal: double declaration of a 
				// or no new variable if a was declared elsewhere
\end{lstlisting}

短变量声明只能出现在函数内,在一些环境中比如 \lstinline|"if", "for", "switch"| 语句的初始器中,可以用来声明局部临时变量。


\section{函数声明}
一个函数声明会将一个标识符(即函数名)绑定到函数上。
\begin{lstlisting}[style=EBNF]
FunctionDecl = "func" FunctionName Signature [ FunctionBody ] .
FunctionName = identifier .
FunctionBody = Block .
\end{lstlisting}
如果函数签字声明了结果参数,函数体的语句列表必须使用一个终止语句结尾。
\begin{lstlisting}[style=golang]
func IndexRune(s string, r rune) int {
	for i, c := range s {
		if c == r {
			return i
		}
	}
	// invalid: missing return statement
}
\end{lstlisting}

一个函数声明可能会忽略函数体。这中声明为实现于 Go 以外的函数提供签字,比如汇编函数。
\begin{lstlisting}[style=golang]
func min(x int, y int) int {
	if x < y {
		return x
	}
	return y
}

func flushICache(begin, end uintptr)  // implemented externally
\end{lstlisting} 

\section{方法声明}
一个方法是一个带有接收器的函数。
一个方法声明会将一个标识符,即方法名,绑定到方法中,并使用接收器的基类型与方法进行关联。
\begin{lstlisting}[style=golang]
MethodDecl = "func" Receiver MethodName Signature [ FunctionBody ] .
Receiver   = Parameters .
\end{lstlisting}
通过在方法名前使用额外的参数部分来指定接收器。
该参数部分必须声明为一个非可变参数,该参数为接收器。
参数类型必须是 \lstinline|T| 或 \lstinline|*T| 这两个形式中的一种,\lstinline|T| 为类型。
由 \lstinline|T| 表示的类型为接收器基类型;他不能是一个指针或这接口类型,并且必须定义在和方法所属的同一个包内。
方法被称为绑定到一个基类型并且方法名只在类型 \lstinline|T| 或 \lstinline|*T| 的选择器中可见。

在方法签字中,非空白接收器标识符必须是独一无二的。
如果方法体内没有引用接收器的值,则可以省略接收器标识符。
同样的规则适用于函数和方法的参数。

对于基类型,绑定到该基类型中的非空白方法名必须是独一无二的。
如果基类型是一个结构体类型,非空白方法和字段名必须不同。

给定类型 \lstinline|Point|,以下声明:
\begin{lstlisting}[style=golang]
func (p *Point) Length() float64 {
	return math.Sqrt(p.x * p.x + p.y * p.y)
}

func (p *Point) Scale(factor float64) {
	p.x *= factor
	p.y *= factor
}
\end{lstlisting}
会使用接收器类型 \lstinline|*Point| 将方法 \lstinline|Length| 和 \lstinline|Scale| 绑定到基类型 \lstinline|Point|。

方法的类型为使用接收器作为第一个参数的函数的类型。比如,方法 \lstinline|Scale| 具有类型:
\begin{lstlisting}[style=golang]
func(p *Point, factor float64)
\end{lstlisting}
但是,用这种方式声明的函数并不是一个方法。



















