% !TeX root = Main.tex

\chapter{变量}
一个变量时用来保存值的存储位置。
变量所允许的值有其类型来决定。

变量声明或者用作函数参数和结果时,以及函数声明的签字或者函数字面量都会为命名变量保留存储空间。
通过内置函数 \lstinline|new| 或者获取符合字面量的地址,则会在运行时为变量申请变量。
这种匿名变量(可以隐式的)通过指针间接寻址来访问。% @FIXME:pointer indirection 的翻译待商榷。

数组,切片,和结构体类型的结构化变量拥有可以独立寻址的元素和字段。
每一个这种元素都表现为一个变量。 

变量声明时给出的类型,使用 \lstinline|new| 调用或者复合字面值亦或结构化变量的元素类型为变量的静态类型(简称为类型)。
接口类型变量具有不同的动态类型,其类型为运行时所赋值给变量的值的具体类型
(除非值为预声明标识符 \lstinline|nil|,这时候没有类型)。
在执行期间动态类型可能会不同,但是存储在接口变量中的值,永远可以赋值给变量的静态类型。
\begin{lstlisting}[style=golang]
var x interface{}  // x is nil and has static type interface{}
var v *T           // v has value nil, static type *T
x = 42             // x has value 42 and dynamic type int
x = v              // x has value (*T)(nil) and dynamic type *T
\end{lstlisting}
一个变量的值可以通过参考表达式中的变量来获取,其值为赋值给变量的最新值。
如果变量还未被赋予一个值,则值为其类型的零值。