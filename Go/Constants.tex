% !TeX root = Main.tex

\chapter{常量}
有 \emph{布尔常量,\rune{}常量,整型常量,浮点常量,复数常量和字符串常量}。
\rune{},整型,浮点,和复数常量通常为\emph{数值}常量。

一个常量通过\rune{}, 整型,浮点,虚数或字符串字面量表示,
也可以通过一个表示常量的标识符,常量表达式,结果为常量的转换来表示,
或者使用一些内置函数,比如适用于任何值的 \lstinline|unsafe.Sizeof| 或者适用于部分表达式的 \lstinline|cap| 或 \lstinline|len|,\lstinline|real| 和 \lstinline|imag| 适用于复数常量,
\lstinline|complex| 适用于数值常量。
布尔真值由预声明常量 \lstinline|true| 和 \lstinline|false| 来表示。
预声明标识符 \lstinline|iota| 则表示一个整型常量。

一般来说,复数常量是常量表达式的一种形式,将在相应的章节中讨论。

% 语句可能不同
数值常量表示任意精度的确切值,并且不会导致溢出。
因此没有常量可以表示 IEEE-754 的负零,无穷,非数值。
常量既可以具有类型也可是无类型的。字面常量,\lstinline|true|、\lstinline|false|、 \lstinline|iota| 以及一些只包含无类型常量操作数的常量表达式属于无类型常量。

一个常量可以通过常量声明或者转换显式的给出其类型,也可以在使用变量声明或赋值亦或表达式中的一个操作数时隐式的给出类型。
如果常量值无法表示相应类型的值,则视为错误。

一个无类型常量具有\emph{默认类型},该类型为常量在上下文环境中隐式转换的所需值的类型,
比如,没有显式类型的短变量声明 \lstinline|i := 0|。
一个无类型常量的默认类型可以为 \lstinline|bool, rune, int, float64, complex128| 或者 \lstinline|string|,这取决于其是否是相应类型的常量。

实现上的约束:尽管数值常量在语言中具有任意的精度,但是编译器可能会使用具有受限精度的内部表示。
也就是说,每一个实现必须:
\begin{itemize}
\item 至少使用 256 个 bit 来表示整型常量。
\item 浮点常量,包括复数常量的各分部的尾数部分至少使用 256 bits 来表示,有符二进制指数则至少 16 bits
\item 如果无法精确的表示整型常量,则视为错误。
\item 如果由于溢出而无法表示一个浮点或复数常量,则视为出错。
\item 如果受限于精度而无法表示一个浮点或复数常量,则取最接近的可表示的常量。
\end{itemize}
这些要求同时适用于字面常量和常量表达式的计算结果。